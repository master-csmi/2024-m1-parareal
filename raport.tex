\documentclass{article}
\usepackage{graphicx}
\usepackage{lipsum}

\title{Parareal Algorithm}
\author{Oussama  BOUHENNICHE, 
Narimane ZAOUACHE}
\date{\today}

\begin{document}

\maketitle

\begin{abstract}
    Solving time-de­pendent partial differe­ntial equations (PDEs) numerically is 
    a significant task in computational domains. Traditional methods can be­ 
    time-consuming, especially for long simulations or comple­x systems. 
    The Parareal algorithm \cite{lions2001resolution} offe­rs an efficient parallel solution to acce­lerate PDE solving.

    In this project, we aim to implement the Parareal algorithm in feel++ framework in parallel and in C++. 
    By integrating the Parareal algorithm with Feel++, 
    we can leverage the advanced numerical methods and parallel computing 
     capabilities of both frameworks to tackle challenging time-dependent PDE problems.
    \end{abstract}

\section{Introduction}

 
 \section{Methodology}
 \subsection{Implementing the Lorenz model}


 The Lorenz system\cite{lorenz1963deterministic}, introduced by meteorologist Edward Lorenz in 1963, 
 is a set of nonlinear ordinary differential equations describing the behavior of a dynamic system. 
 This system is used as a mathematical model to study phenomena such as convection and turbulence, 
 as well as in other areas such as population dynamics, chaos theory, and control theory.



 The Lorenz system is described by the following three equations:

 \[
\left\{
\begin{array}{ccc}
\frac{dx}{dt} = \sigma(y - x) \\
\frac{dy}{dt} = x(\rho - z) - y \\
\frac{dz}{dt} = xy - \beta z
\end{array}
\right.
\]

Where $x$, $y$, and $z$ represent the state variables of the system, 
$t$ is time, and $\sigma$, $\rho$, and $\beta$ are positive real parameters.

Edward Lorenz discovered that, for the parameter values 
$\sigma = 10$, $b = 8/3$, and $\rho = 28$, a large set of solutions are attracted to a butterfly shaped set (called the Lorenz attractor). 
The trajectory seems to randomly jump betwen the two wings of the butterfly. 
The behavior exhibited by the system is called "chaos", while this type of attractor is called a "strange attractor". 

The numerical solution of the Lorenz system involves using numerical methods to approximate the trajectory of the state variables over time. 
Two commonly used methods are the fourth-order Runge-Kutta (RK4) method and the Euler method.

\subsubsection{First-order method: Euler Method}
The Euler method is an iterative method that involves approximating the temporal derivatives 
of the state variables using finite differences and using these approximations to update the values of the state variables at each time step. 

\subsubsection{Fourth-order method: Runge-Kutta (RK4) Method}
The fourth-order Runge-Kutta (RK4) method is an iterative method that involves discretizing 
time and using iterations to compute the values of the state variables at each time step. 
This method uses four successive approximations to improve the accuracy of the solution. 


\section{Results and Discussion}


\section{Conclusion}

\bibliographystyle{plain}
\bibliography{./References}
\end{document}