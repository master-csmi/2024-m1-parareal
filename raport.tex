\documentclass{article}
\usepackage{graphicx}
\usepackage{lipsum}

\title{Report on the Lorenz System and Parareal}
\author{Oussama  BOUHENNICHE, 
Narimane ZAOUACHE}
\date{\today}

\begin{document}

\maketitle

\begin{abstract}
The Lorenz system is a set of ordinary differential equations that models the behavior of a moving fluid, 
particularly the phenomena of convection and turbulence. This report explores the Lorenz system, presents its mathematical model, 
and discusses how it can be numerically solved. Additionally, this report examines the approach of parallelizing the computation of the Lorenz system trajectory, 
illustrated using a Python code example.
\end{abstract}

\section{Introduction}
The Lorenz system, introduced by meteorologist Edward Lorenz in 1963, 
is a set of nonlinear ordinary differential equations describing the behavior of a moving fluid. 
This system is widely used as a mathematical model to study phenomena such as convection and turbulence in fluids,
 as well as in various other areas including population dynamics, chaos theory, and control theory.



 



\section{Conclusion}

\end{document}